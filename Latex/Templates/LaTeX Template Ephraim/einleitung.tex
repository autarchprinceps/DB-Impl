\chapter{Einleitung}

\section{Aufgabenstellung}
Im Rahmen dieser Studienarbeit sollte ein System erstellt werden, mit dem der Lernfortschritt und
das Verständnis  der Studenten an der \gls{HDM} für die in der Vorlesung vermittelten Inhalte
abgeprüft werden sollte. Dieses System sollte \gls{LeMon} heißen. Dabei sollte ein Dozent eine Art
Fragebogen für Studenten vorbereiten, der in den ersten Minuten der Vorlesung von den Studenten
durchgearbeitet werden sollte. Die Studentenantworten sollten soweit möglich automatisiert
ausgewertet oder zumindest anschaulich in einer Übersicht dargestellt werden, die anschließend am
Beamer gezeigt und durch die Lehrkraft besprochen werden sollte. In diesem Schritt sollten sich dann
Unklarheiten der Studenten herauskristalisieren. Der Dozent sollte dabei merken, welche Themen einer
Wiederholung gedürfen und welche nicht.

Das System sollte dabei den kompletten Prozess von der Erstellung einzelner Übungsblätter durch den
Dozenten, das Management der Fragen für die Übungsbläter, sowie das Freischalten der Übungsblätter
zur entsprechenden Zeit beinhalten.
Desweiteren sollte der Dozent den Studenten auf einfache Art und Weise das elektronische Übungsblatt
zugänglich machen. Hierzu sollte bevorzugt ein QR-Code eingesetzt werden, den die Studenten mit
ihrem Smartphone und einer entsprechenden App sofort einlesen und zum Übungsblatt gelangen konnten.
\newline Am Ende sollte der Dozent auf einfache Weise die Möglichkeit haben, die Ergebnisse wenige
Sekunden nach Ablauf der Zeit anschaulich angezeigt zu bekommen und diese mit den Studenten
durchsprechen.

Zur Umsetzung dieses Projektes sollte eng mit einer Gruppe mit einem ähnlichen Projekt
zusammengearbeitet werden, die jedoch eine etwas andere Art von Übungsblättern für den Einsatz von
Aufgaben innerhalb oder auch außerhalb der Vorlesung erstellen sollte. Soweit möglich sollten dabei
Synergien genutzt und eine gemeinsame Oberfläche für beide Projekte erstellt werden.


\section{Zielsetzung}
Diese Studienarbeit sollte nicht nur zur "`Absolvierung"' der Studienarbeit dienen, sondern auch den
Studenten einen praktischen Nutzen durch das Erlernen und Vertiefen von Wissen im Bereich moderner
Technologien bieten. Zusätzlich war ein klares Ziel dieser Studienarbeit, dass das Endprodukt nicht
nur irgend einen theoretischen Nutzen hat oder ein Proof of Concept sein sollte, sondern
anschließend praktische Verwendung findet.
 

% Hier könnte eine Einleitung stehen \emph{Hervorgehoben} \cite{LiteraturEintrag1}
% 
% \section{Überschrift abc}
% Abcdef ghijklmno pqrstuvw xyz äöü\ldots \gls{DHBW}
%  
% \section{Überschrift abcd}\label{sec:Ueberschriftabcd}
% Abcdef ghijklmno pqrstuvw xyz ÄÖÜ\ldots 
% 
% \prettyref{sec:Ueberschriftabcd}
% 
% 
% \begin{figure} [!htb]
% 	\begin{center}
% 		  \begin{tabular}{@{}r@{}}
% 		{\includegraphics[width=14cm]{images/dhbw-Logo.png}}\\
% 		\footnotesize\sffamily\textbf{Quelle:} DHBW-Homepage
% 	  		   \cite{LiteraturEintrag1}
%  	 	 \end{tabular} 	
% 		\caption{Das DHBW-Logo}
% 		\label{fig:DHBWLogo}
% 	\end{center} 
% \end{figure}