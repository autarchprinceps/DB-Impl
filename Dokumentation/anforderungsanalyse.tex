\chapter{Anforderungsanalyse} 
\section{Use-Cases} \label{sec:Use-Cases}

% %Ideen für Spacing:
% %\itemsep -5pt
% %Vorlage:
% \begin{table}[h]
% 	\begin{tabular}{|p{3cm}|p{11.06cm}|}
% 	\hline
% 		\multicolumn{2}{|l|}{\textbf{UC Nr.} }   \\ \hline
% 		\textbf{Name}                 &         \\ \hline
% 		\textbf{Ziel im Kontext}      &         \\ \hline
% 		\textbf{Akteure}              &         \\ \hline
% 		\textbf{Trigger}              &         \\ \hline
% 		\textbf{Essenzielle Schritte} & 
% 			\begin{enumerate}
% 			  \item sdf
% 			  \item 
% 			\end{enumerate}
% 		\\ \hline
% 		\textbf{Erweiterungen} 		  &         \\ \hline
% 	\end{tabular}
% \end{table}\FloatBarrier
% 
% \begin{table}[h]
% 	\begin{tabular}{|p{3cm}|p{11.06cm}|}
% 	\hline
% 		\textbf{Zugehöriger Use Case}                 &         \\ \hline
% 		\textbf{Beteiligte Tabellen}      &         \\ \hline
% 		\textbf{Vorbedingung}              &         \\ \hline
% 		\textbf{Ablauf}              &   
% 			\begin{enumerate}
% 			  \item sdf
% 			  \item 
% 			\end{enumerate}
% 		\\ \hline
% 		\textbf{Erweiterungen}              &         \\ \hline
% 	\end{tabular}
% \end{table}\FloatBarrier

\begin{table}[h] 
\subsection{UC1: Frage mit Musterlösungen anlegen}\label{uc:UC1}
	\begin{tabular}{|p{3cm}|p{11.06cm}|}
	\hline
		\multicolumn{2}{|l|}{\textbf{UC Nr.: UC1} }   \\ \hline
		\textbf{Name}                 &      Frage mit Musterlösungen anlegen   \\ \hline
		\textbf{Ziel im Kontext}      &      Dozent legt eine neue Frage mit ihrer Musterlösung an   \\ \hline
		\textbf{Akteure}              &      Dozent (D.)   \\ \hline
		\textbf{Trigger}              &      D. klickt auf "`Frage anlegen"'   \\ \hline
		\textbf{Essenzielle Schritte} & 			
			\begin{enumerate}
			  \item D. befindet sich im Admin Bereich
			  \item D. klickt auf „Frage anlegen“
			  \item D. sieht Web Interface zum Frage erstellen
			  \item	D. wählt den Fragentyp aus einer Auswahlmöglichkeit aus
			  \item D. wählt die Kategorie für die Frage aus einer Auswahlmöglichkeit aus
			  \item D. gibt Frage in das dazu passende Feld ein
			  \item D. gibt die zur Frage passende/n Musterlösung/en in das/die passende/n Feld/er ein
			  \item D. klickt abschließend auf „Abschicken“
			\end{enumerate}			
		\\ \hline
		\textbf{Erweiterungen} 		  &   4a. D. gibt an wie viele Antworten für Aufzählaufgaben gefordert werden     \\ \hline
	\end{tabular}
	\caption{Use Case 1: Frage mit Musterlösungen anlegen}
\end{table}\FloatBarrier




\begin{table}
\subsection{UC2: Kategorie anlegen}\label{uc:UC2}
	\begin{tabular}{|p{3cm}|p{11.06cm}|}
	\hline
		\multicolumn{2}{|l|}{\textbf{UC Nr.: UC2} }   \\ \hline
		\textbf{Name}                 &     Kategorie anlegen   \\ \hline
		\textbf{Ziel im Kontext}      &     Dozent legt eine neue Kategorie an   \\ \hline
		\textbf{Akteure}              &     Dozent (D.)    \\ \hline
		\textbf{Trigger}              &     D. klickt auf „Kategorie anlegen“    \\ \hline
		\textbf{Essenzielle Schritte} & 
			\begin{enumerate} 
			  \item D. befindet sich im Admin Bereich
			  \item D. klickt auf „Kategorie anlegen“
			  \item D. sieht Web Interface zum Kategorien erstellen
			  \item D. gibt den Namen der Kategorie in das dazu passende Feld ein
			  \item	D. klickt abschließend auf den „Plus“-Button
			\end{enumerate}
		\\ \hline
		\textbf{Erweiterungen} 		  &         \\ \hline
	\end{tabular}
	\caption{Use Case 2: Kategorie anlegen}
\end{table}\FloatBarrier



\begin{table}[h]
\subsection{UC3: Vorlesung anlegen}\label{uc:UC3}
	\begin{tabular}{|p{3cm}|p{11.06cm}|}
	\hline
		\multicolumn{2}{|l|}{\textbf{UC Nr.: UC3} }   \\ \hline
		\textbf{Name}                 &     Vorlesung anlegen    \\ \hline
		\textbf{Ziel im Kontext}      &     Dozent legt eine neue Vorlesung an   \\ \hline
		\textbf{Akteure}              &     Dozent (D.)    \\ \hline
		\textbf{Trigger}              &     D. klickt auf „Vorlesung anlegen“    \\ \hline
		\textbf{Essenzielle Schritte} & 
			\begin{enumerate}
			  \item D. befindet sich im Admin Bereich
			  \item D. klickt auf „Vorlesung anlegen“
			  \item D. sieht Web Interface zum Vorlesungen anlegen
			  \item D. gibt den Namen der Vorlesung in das dazu passende Feld ein
			  \item D. klickt abschließend auf den „Plus“-Button
			\end{enumerate}
		\\ \hline
		\textbf{Erweiterungen} 		  &         \\ \hline
	\end{tabular}
	\caption{Use Case 3: Vorlesung anlegen}
\end{table}\FloatBarrier



\begin{table}[h]
\subsection{UC4: Arbeitsblatt anlegen}\label{uc:UC4}
	\begin{tabular}{|p{3cm}|p{11.06cm}|}
	\hline
		\multicolumn{2}{|l|}{\textbf{UC Nr.: UC4} }   \\ \hline
		\textbf{Name}                 &     Arbeitsblatt anlegen    \\ \hline
		\textbf{Ziel im Kontext}      &     Dozent legt ein neues Arbeitsblatt an    \\ \hline
		\textbf{Akteure}              &     Dozent (D.)    \\ \hline
		\textbf{Trigger}              &     D. klickt auf „Arbeitsblatt anlegen“    \\ \hline
		\textbf{Essenzielle Schritte} & 
			\begin{enumerate}
			  \item D. befindet sich im Admin Bereich
			  \item D. klickt auf „Arbeitsblatt anlegen“
			  \item D. sieht Web Interface zum Erstellen eines Arbeitsblattes
			  \item D. gibt den Namen des Arbeitsblattes in das dazu passende Feld ein
			  \item D. erhält Übersicht aller Fragen mit Checkbox zum Auswählen mit Filtermöglichkeit
			  \item D. sieht eine extra Tabelle mit den bisher ausgewählten Fragen
			  \item D. klickt nach Hinzufügen aller gewünschten Fragen auf „Arbeitsblatt anlegen“, wenn alle gewünschten Fragen hinzugefügt wurden
			\end{enumerate}
		\\ \hline
		\textbf{Erweiterungen} 		  &   6a. D. kann die Reihenfolge der Fragen verändern oder sie wieder entfernen      \\ \hline
	\end{tabular}
	\caption{Use Case 4: Arbeitsblatt anlegen}
\end{table}\FloatBarrier




\begin{table}[h]
\subsection{UC5: Antworten abschicken}\label{uc:UC5}
	\begin{tabular}{|p{3cm}|p{11.06cm}|}
	\hline
		\multicolumn{2}{|l|}{\textbf{UC Nr.: UC5} }   \\ \hline
		\textbf{Name}                 &     Antworten abschicken    \\ \hline
		\textbf{Ziel im Kontext}      &     Studenten schicken Antworten ab    \\ \hline
		\textbf{Akteure}              &     Student (S)    \\ \hline
		\textbf{Trigger}              &     S. klickt auf „Antworten abschicken“    \\ \hline
		\textbf{Essenzielle Schritte} & 
			\begin{enumerate}
			  \item S. befindet sich im Studenten-Bereich
			  \item S. klickt auf „Antworten abschicken“
			  \item S. sieht Bestätigungsdialog
			  \item S. klickt auf „OK“
			  \item S. sieht Bestätigungsdialog
			\end{enumerate}
		\\ \hline
		\textbf{Erweiterungen} 		  &   4a. S. klickt auf "`Abbrechen"' und schickt somit seine Antworten noch nicht ab     \\ \hline
	\end{tabular}
	\caption{Use Case 5: Antworten abschicken}
\end{table}\FloatBarrier




\begin{table}[h]
\subsection{UC6: Auswertung des Übungsblattes}\label{uc:UC6}
	\begin{tabular}{|p{3cm}|p{11.06cm}|}
	\hline
		\multicolumn{2}{|l|}{\textbf{UC Nr.: UC6} }   \\ \hline
		\textbf{Name}                 &     Auswertung des Arbeitsblattes    \\ \hline
		\textbf{Ziel im Kontext}      &     Dozent wertet Arbeitsblatt aus   \\ \hline
		\textbf{Akteure}              &     Dozent (D)    \\ \hline
		\textbf{Trigger}              &     D. klickt auf „Auswertung“    \\ \hline
		\textbf{Essenzielle Schritte} & 
			\begin{enumerate}
			  \item D. befindet sich im Admin-Bereich
			  \item D. klickt auf „Auswertung“
			  \item D. sieht Übersicht der verschiedenen Vorlesungen und Arbeitsblättern
			\end{enumerate}
		\\ \hline
		\textbf{Erweiterungen} 		  &         \\ \hline
	\end{tabular}
	\caption{Use Case 6: Auswertung des Übungsblattes}
\end{table}\FloatBarrier

 

\section{Allgemeine Anforderungen}
Die an das Projekt gestellten Anforderungen gliedern sich in zwei Gebiete:
Technologische Anforderungen und allgemeine Anforderungen, die an die
Applikation gestellt werden. Soweit möglich sollte bei der Datenhaltung
Synergie-Effekte mit dem Projekt "`CCKE"' benutzt werden. Ebenso sollte der
Bereich für Lehrbeauftragte für die Erstellung und Verwaltung und Präsentation
der Ergebnisse weit möglichst für beide Projekte einheitlich gehalten werden.

\section{Technologische Anforderungen}
Die zu entwickelnde Applikation sollte auf Basis von Webtechnologien aufgebaut
werden. Desweiteren sollte das Backend für Lehrbeauftragte sowie die Auswertung
der Ergebnisse an einem Notebook bedient werden können. Die Oberfläche für die
Auswertung der Antworten sollte dabei so optimiert sein, dass die Ergebnisse
ohne größeren Aufwand mittels einem im Raum vorhandenen Beamer präsentiert
werden können.  \newline
Die Oberfläche für die Eingabe der Antworten der Studenten sollte dabei für
mobile Geräte optimiert werden, was insbesondere Smartphones und Tablets
beinhalten sollte. Die Optimierung für mobile Geräte sollte die Bedienbarkeit
für verbreitete mobile Platformen sicherstellen. \newline
Die gesamte Datenhaltung der Applikation sollte mit Hilfe einer frei verfügbaren
Datenbank realisiert werden. 

\section{Eingesetzte Technologien}
Bei der Auswahl der Technologien wurde insbesondere Wert darauf gelegt, dass die
Grundlagen für diese den Studenten bereits bekannt sind. Allerdings sollten auch
neue Aspekte unter  zuhilfenahme moderner Ansätze und Frameworks erlernt und so
die Grundlage für eine spätere Erweiterung des Projekts durch nachfolgende
Gruppen sichergestellt werden.

Nachfolgend eine Auflistung der eingesetzten Technologien:
\begin{singlespacing}
  \begin{multicols}{3}
\begin{itemize}
  \item HTML
  \item CSS
  \item JavaScript
  \item AJAX
  \item PHP
  \item SQL
\end{itemize}
\end{multicols}
\end{singlespacing}